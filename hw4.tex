\documentclass[a4paper,12pt]{article} % This defines the style of your paper

\usepackage[top = 2.5cm, bottom = 2.5cm, left = 2.5cm, right = 2.5cm]{geometry} 

\usepackage[T1]{fontenc}
\usepackage[utf8]{inputenc}

\usepackage{multirow} % Multirow is for tables with multiple rows within one cell.
\usepackage{booktabs} % For even nicer tables.

\usepackage{graphicx} 
\usepackage{amsfonts}
\usepackage{amsmath}
\usepackage{algorithm}
\usepackage{algpseudocode}
\usepackage{algorithmicx}

\usepackage{setspace}
\setlength{\parindent}{0in}

\usepackage{float}

\usepackage{fancyhdr}


\pagestyle{fancy} % With this command we can customize the header style.

\fancyhf{} % This makes sure we do not have other information in our header or footer.

\rhead{\footnotesize J. Klepl}

\cfoot{\footnotesize \thepage} 

\begin{document}

\thispagestyle{empty} % This command disables the header on the first page. 

\begin{center}
    {\Large \bf Homework 4 - reg}
    \vspace{2mm}

    {\bf Jiří Klepl}

\end{center}

\vspace{0.4cm}


\begin{quote}
    Nechť $B$ je jazyk přijímaný konečným automatem a nechť $A$ je jazyk pro nějž $A \leq_m B$. Je $A$ regulární? Proč?
\end{quote}

Vyvrátíme sporem.

\subsection*{Předpoklady sporu}

Mějme regulární jazyk $B = \mathbf{a}+$ nad abecedou $\Sigma^\star = \{a, b\}$, jazyk $A$ libovolný rozhodnutelný a $E$ jeho enumerátor, jehož výstup je lexikograficky uspořádán.

\subsection*{Definice speciálního turingova stroje přijímajícího jazyk $A$}

Definujme turingův stroj $M$ přijímající $A$ následovně:
\begin{algorithm}
    \caption{$M$ se slovem $y$ na pásce}
    \begin{algorithmic}[1]
        \State $E \gets \text{Enumerátor jazyka }  A$
        \State $W \gets \mathbf{a}$
        \While{$!E.empty()$}
        \State $x \gets E.front()$
        \State $E.pop()$
        \If{$x < y$}
        \State $W \gets W \cdot \mathbf{a}$
        \ElsIf{$x == y$}
        \State \Return{$\text{Přijmi vstup a na pásce zanech } W$}
        \Else
        \State \Return Odmítni vstup
        \EndIf
        \EndWhile
    \end{algorithmic}
\end{algorithm}

Toto lze samozřejmě jednoduše za pomoci simulace emulátoru na jedné pásce za současného zaplňování druhé pásky symboly $a$ a následovaného odmazáváním prvního vyenumerovaného slova, pokud je lexikograficky menší než vstup. Pokud je toto slovo rovné vstupu, necháme na výstupu pouze obsah druhé pásky. Pokud je vyšší, odmítneme vstup.

Tento postup je jistě konečný, neboť je jen konečně mnoho lexikograficky menších slov, než je vstup. Tedy turingův stroj $M$ vždy skončí.

\subsection*{Převod}

Poté definujeme $f$ pomocí turingova stroje $M$, poté: $B = f(A)$ a tedy $A \leq_m B$.

\subsection*{Spor}

$A \leq_m B$ a $B$ je regulární. Mějme tedy $A = \{\mathbf{a}^i\mathbf{b}^i | i \in \mathbb{N}\}$ porušující regulární pumping lemma a to nám dává spor.

\end{document}
