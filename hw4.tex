\documentclass[a4paper,12pt]{article} % This defines the style of your paper

\usepackage[top = 2.5cm, bottom = 2.5cm, left = 2.5cm, right = 2.5cm]{geometry} 

\usepackage[T1]{fontenc}
\usepackage[utf8]{inputenc}

\usepackage{multirow} % Multirow is for tables with multiple rows within one cell.
\usepackage{booktabs} % For even nicer tables.

\usepackage{graphicx} 
\usepackage{amsfonts}
\usepackage{amsmath}
\usepackage{algorithm}
\usepackage{algpseudocode}
\usepackage{algorithmicx}

\usepackage{setspace}
\setlength{\parindent}{0in}

\usepackage{float}

\usepackage{fancyhdr}


\pagestyle{fancy} % With this command we can customize the header style.

\fancyhf{} % This makes sure we do not have other information in our header or footer.

\rhead{\footnotesize J. Klepl}

\cfoot{\footnotesize \thepage} 

\begin{document}

\thispagestyle{empty} % This command disables the header on the first page. 

\begin{center}
    {\Large \bf Homework 4 - reg}
    \vspace{2mm}

    {\bf Jiří Klepl}

\end{center}

\vspace{0.4cm}


\begin{quote}
    Nechť $B$ je jazyk přijímaný konečným automatem a nechť $A$ je jazyk pro nějž $A \leq_m B$. Je $A$ regulární? Proč?
\end{quote}

Ukážeme protipříklad naznačované implikace: nechť $B$ je regulární jazyk definován regulárním výrazem $\mathbf{a}+$ nad abecedou $\{a\}$ a $A$ je libovolný rozhodnutelný jazyk, který není regulární, takový jistě existuje. Nechť $M$ je automat rozhodující $A$.

Definujeme stroj $M'$, který simuluje práci $M$, ale má vlastní výstup. Na výstupu (pásce), pokud $M$ přijal, $M'$ zanechá $\mathbf{a}^k$, kde $k$ je počet kroků, jež $M$ provedl ($M'$ si čárkuje kroky simulace), jinak ponechá pásku prázdnou. Lze rozmyslet, že $k$ je vždy dobře definované konečné číslo.

$f_{M'}$ převádí problém jazyka $A$ na problém jazyka $B$ a tedy $A \leq_m B$, přestože $A$ regulární není.

\end{document}
