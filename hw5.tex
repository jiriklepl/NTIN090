\documentclass[a4paper,12pt]{article} % This defines the style of your paper

\usepackage[top = 2.5cm, bottom = 2.5cm, left = 2.5cm, right = 2.5cm]{geometry} 

\usepackage[T1]{fontenc}
\usepackage[utf8]{inputenc}

\usepackage{multirow} % Multirow is for tables with multiple rows within one cell.
\usepackage{booktabs} % For even nicer tables.

\usepackage{graphicx}
\usepackage{amsmath}
\usepackage{amsfonts}

\usepackage{setspace}
\setlength{\parindent}{0in}

\usepackage{float}

\usepackage{fancyhdr}


\pagestyle{fancy} % With this command we can customize the header style.

\fancyhf{} % This makes sure we do not have other information in our header or footer.

\rhead{\footnotesize J. Klepl}

\cfoot{\footnotesize \thepage} 

\begin{document}

\thispagestyle{empty} % This command disables the header on the first page. 

\begin{center}
    {\Large \bf Homework 5 - size}
    \vspace{2mm}

    {\bf Jiří Klepl}

\end{center}

\vspace{0.4cm}


\begin{quote}
    Je jazyk $\text{Size} = \{\left<M,k\right> \mid \left\vert L(M) \right\vert \geq k\}$ rozhodnutelný? Je částečně rozhodnutelný?
\end{quote}

\subsection*{Částečná rozhodnutelnost}

Pokud $L(M) \geq k$, pak určitě existuje slovo $m \in L(M)$, že: $\{x \mid x \in L(M): x \leq m\} = k$. Tedy stačí ověřit konečný počet slov, aby se potvrdilo, že $\left<M, k\right> \in \text{Size}$, neboť existuje pouze konečně mnoho slov lexikograficky menších $m$.

Tedy existuje turingův stroj, který tato slova projde v konečném počtu kroků a rozhodne o přijetí. Jazyk je tedy (alespoň) \textbf{částečně rozhodnutelný}.

\subsection*{Rozhodnutelnost}

$m$-úplný jazyk $\textbf{NE}$ (not empty), který rozhoduje neprázdnost jazyků turingových strojů, lze $m$-převést na jazyk $\text{Size}$ (jednoduchým řetězením: $f(M) = \left<M , \mathbf{1} \right>$), tedy jazyk $\text{Size}$ jistě není \textbf{rozhodnutelný}.

\end{document}
