\documentclass[a4paper,12pt]{article} % This defines the style of your paper

\usepackage[top = 2.5cm, bottom = 2.5cm, left = 2.5cm, right = 2.5cm]{geometry} 

\usepackage[T1]{fontenc}
\usepackage[utf8]{inputenc}

\usepackage{multirow} % Multirow is for tables with multiple rows within one cell.
\usepackage{booktabs} % For even nicer tables.

\usepackage{graphicx}
\usepackage{amsmath}
\usepackage{amsfonts}

\usepackage{setspace}
\setlength{\parindent}{0in}

\usepackage{float}

\usepackage{fancyhdr}


\pagestyle{fancy} % With this command we can customize the header style.

\fancyhf{} % This makes sure we do not have other information in our header or footer.

\rhead{\footnotesize J. Klepl}

\cfoot{\footnotesize \thepage} 

\begin{document}

\begin{quote}
    O kterých inkluzích mezi následujícími dvojicemi tříd jste schopni dokázat, že platí a o kterých, že neplatí. Za každý dokázaný vztah je jeden bod (do požadovaného počtu bodů započítáno za 3).

    \begin{tabular}{l l l}
        
        1. & $TIME(2^n)$ & $NSPACE(\sqrt{n})$ \\
        2. & $NSPACE((\log n)^3)$ & $SPACE(n)$ \\
        3. & $NTIME(n^3)$ & $SPACE(n^6)$ \\
    \end{tabular}
\end{quote}

\section*{Příklad 1}

Uvažujme problém rozhodnutelný $M = (\Sigma, Q, q_0, \delta, F)$, nedeterministickým TS, v prostoru $NSPACE(\sqrt{n})$. Takový má $|\Sigma|^{\sqrt{n}} \cdot |Q| \cdot \sqrt{n}$ možných displayů, tuto množinu nazvěme $D$ a definujme přestupy mezi displayi $T$ podle $\delta$. Definujeme graf $G = (D, T)$.

Potom úpravou $|\Sigma|^{\sqrt{n}} \cdot |Q| \cdot \sqrt{n} \in 2^{O(\sqrt{n})} \subseteq O(2^n)$ a tedy graf $G$ má $O(2^n)$ vrcholů. Hledáme tedy za pomoci deterministického turingova stroje existenci sledu mezi iniciálním displayem a nějakým koncovým (BÚNO právě jediným koncovým), což jistě v čase $TIME(2^n)$ stihneme.

Nemožnost rozhodnout libovolný problém $Time(2^n)$ v prostoru $NSPACE(\sqrt{n})$ je zřejmá.

Platí tedy:
$$TIME(2^n) \supset NSPACE(\sqrt{n})$$

\section*{Příklad 2}

\section*{Příklad 3}

Uvažujme problém rozhodnutelný $M = (\Sigma, Q, q_0, \delta, F)$, nedeterministickým TS, v čase $NTIME(n^3)$ (tedy i v prostoru $NSPACE(n^3)$). Tedy v hloubce $n^3$ alespoň jedna z větví výpočtu $M$ přijme, každá dobře definovaná nějakým indexem z $(|\Sigma| \cdot |Q| \cdot 3)^{n^3}$ (na základě jednotlivých rozhodnutí), což lze zapsat na vhodné abecedě $\Sigma'$ za pomoci $O(n^3)$ symbolů. Výpočet $M$ lze odsimulovat deterministickým turingovým strojem v prostoru $SPACE(n^3)$ za používání návodu na rozhodnutí použité větve iterovaném taktéž ve $SPACE(n^3)$.

Mějme pak problém výpočtu $k^{n^6}$, což je problém řešitelný ve $SPACE(n^6)$, ale jistě ne v $NTIME(n^3)$, neboť $NSPACE(n^3)$ není dostačující a využití pásky je dolním odhadem časové složitosti.

Platí tedy:
$$NTIME(n^3) \subset SPACE(n^6)$$

\end{document}
